\section{TVS}

The input structures for the analysis are described in a format
called TVS (Three Valued Structure). For example, the TVS for
input structure used in the analysis of the reverse function is
given in \figref{ManReverseTVS}. A TVS file name should end with
the extension '.tvs'. The syntax of a TVS file is given in
\figref{TVSSyntax}.

\begin{figure}
\framebox{
\begin{minipage}{1in}
\begin{tabbing}
\param{tvs} ::= \param{structure}$^*$\\
\param{structure} ::= \param{universe} \param{predicates}\\
\param{universe} ::= \textbf{\%n}\ \bassign\ \blcb\ \csep{\param{node}}\ \brcb\\
\param{predicates} ::= \textbf{\%p}\ \bassign\ \blcb\ \param{predicate}$^*$\ \brcb\\
\param{predicate} ::\=\+= \ppred\ \bassign\ \param{kleene} /* Nullary */\\
$|$ \ppred\ \bassign\ \blcb\ \csep{(\param{node} \lb\param{value}\rb)}\ \brcb\ /* Unary */\\
$|$ \ppred\ \bassign\ \blcb\
\csep{(\param{leftnode}\deref\param{rightnode}
\lb\param{value}\rb)}\ \brcb\ /* Binary */\\
$|$ \ppred\ \bassign\ \blcb\ \csep{(\csep{\pid})
\lb\param{value}\rb
}\ \brcb\ /* Arbitrary */\-\\
\param{node} ::=\+\=\ \pid\ \-\\
%%%%%%%%% -significant is no longer supported %%%%%%%%%%
%$|$ \blb\ \csep{\param{node}}\ \brb\\
%$|$ \param{node}\textbf{.}(\textbf{0}$|$\textbf{1})\-\\
%%%%%%%%%%%%%%%%%%%%%%%%%%%%%%%%%%%%%%%%%%%%%%%%%%%%%%%%
\param{value} ::= \textbf{:} \param{kleene}
\end{tabbing}
\end{minipage}
} \caption{\label{Fi:TVSSyntax}The syntax of a TVS file.}
\end{figure}

The value of a predicate defaults to false unless otherwise
specified in the TVS structure. All the node names used in the
predicates must be predefined. All the predicate names used must
be declared in the TVP file. If a node (or a node pair) is
specified without a value, the default of true (1) is taken. TVS
supports the same commenting style as TVP.
