\section{\label{Se:Command}Command line options}
\small
\begin{alltt}
Usage: tvla <program name> [input file] [options] Options:
 -d                      Turns on debug mode.
 -action [f][c]pu[c]b    Determines the order of operations computed
                         by an action. The default is fpucb.
                         f - Focus,  c - Coerce, p - Precondition.
                         u - Update, b - Blur.
 -join [algorithm]       Determines the type of join method to apply.
                         rel  - Relational join.
                         part - Partial join.
                         conc - No abstraction (i.e., blurring) is applied.
 -ms <number>            Limits the number of structures.
 -mm <number>            Limits the number of messages (default=0).
 -save {back|ext|all}    Determines which locations store structures.
                         back - at every back edge (the default).
                         ext  - at every beginning of an extended block.
                         all  - at every program location.
 -noautomatic            Supresses generation of automatic constraints.
 -props <file name>      Can be used to specify a properties file.
 -log <file name>        Creates a log file of the execution.
 -tvs <file name>        Creates a TVS formatted output.
 -dot <file name>        Creates a DOT formatted output.
 -tr:tvs <file name>     Creates a transition relation output in tvs-like format.
 -tr:dot <file name>     Creates a transition relation output in dot format.
 -D<macro name>[(value)] Defines a C preprocessor macro.
 -terse                  Turns off on-line information printouts.
 -nowarnings             Causes all warnings to be ignored.
 -path <directory path>  Can be used to specify a search path.
 -post                   Post order evaluation of actions.
\end{alltt}

\begin{itemize}
\item{\textbf{Analysis engine}}\\
Three different types of engines are available : \textbf{tvla} is
the classic chaotic iteration algorithm and the default one,
\textbf{tvmc} is a multithreading engine that performs a
state-space exploration using a search stack, and \textbf{ddfs} is
a Double-DFS multithreading engine that utilizes Buchi automata.

\item{\textbf{Backward Analysis}}\\
Some analyses need to propagate information in the opposite
direction specified for the CFG edges. To reverse the direction of
the CFG edges use the \textbf{-backward} flag. When this option is
chosen, the input structures are stored in the last location
computed by the topological sorting of the program locations.

\item{\textbf{Order of action evaluation}}\\
The default order of evaluation in the iterative algorithm is
reverse post order. However, when the analysis is very time/space
consuming and you want to see the structures that reach the end of
the analyzed program as soon as possible, use the \textbf{-post}
flag to use post order and get the desired effect.

\item{\textbf{Debugging}}\\
When debugging a new analysis it is useful to see the analysis as
it progresses and not just its final result. Use the \textbf{-d}
flag to see the structures in the different phases of execution.
In debug mode all the consistency rules are printed together with
their dependencies and each time a structure is discarded because
of an irreparable consistency rule breach, the problematic
consistency rule, assignment and structure are shown. Notice that
this mode generates very large PostScript files so you would
probably want to use the \textbf{-ms} flag.

\item{\textbf{Computing the effect of an action}}\\
Sometime you want to try and run the algorithms (Coerce, Focus,
Precondition, Update, Blur) in a different order or quantity than
the default one (Focus, Coerce, Precondition, Update, Coerce,
Blur). Use the \textbf{-action
\param{seq}} flag to control the computation of the action's
effect.  The argument is of the form [f][c]pu[c]b when: f - Focus,
c - Coerce, p - Precondition, u - Update, b -
Blur. % blur is now mandatory

\item{\textbf{Join method}}\\
Three join methods are available. To use the relational analysis
approach where (bounded) structures kept up to isomorphism choose
the \textbf{rel} option. To use the single-structure method use
the \textbf{ind} option. In this approach all the structures in a
CFG node that match with their nullary predicates' values are
merged into a single structure. The option is very useful for
analyses that would otherwise take a very long time and create
many structures. It is worth considering the \textbf{-action
fcpucb} specification when working in single structure mode. A
compromise between the two approaches merges structures that
identify on their nullary abstraction values and set of canonical
node names, thus considering only a partial set of predicates. To
use this option choose the \textbf{part}.  TVLA can perform
analysis without applying abstraction (i.e., blurring) via the
\textbf{conc} option.  This option can yield a
non-terminating analysis and therefore the \textbf{-ms} is usually
needed to force termination.

\item{\textbf{Maximum number of structures}}\\
A complex analysis may take a very long time and especially in
debug time may run forever. To see a partial result, you can limit
the number of structures generated using the \textbf{-ms
\param{number}} flag.

\item{\textbf{Maximum number of messages}}\\
The number of messages reported by the system can be limited. This
can be used to supply a condition that if holds the system stops
(by limiting the number of messages to $1$).

\item{\textbf{Saved locations}}\\
The default behavior of the system is to perform a join, and save
all the structures that reached the program location, only in
every back edge in the control graph (approximately once in every
loop). Saving structures at every program location is the most
efficient in terms of the number of structures generated (use
\textbf{-save all}). However, it is very space consuming. For a
compromise between the two extremes use \textbf{-save ext} which
saves structures only at the beginning of each extended block
(i.e., at every merge point in the control graph).

\item{\textbf{No automatic constraint generation}}\\
Sometimes it is useful to supply all the constraints by hand
without the automatically generated constraints. To do this supply
the \textbf{-noautomatic} flag.

\item{\textbf{Properties file}}\\
A properties file name can be supplied with the \textbf{-props}
option. The file is loaded before the analysis starts and
overrides all other property files. For more information see
section~\ref{Se:PropertyFiles}.

\item{\textbf{Log file}}\\
A log file name can be supplied with \textbf{-log} option. In this
case the majority of the information written to the console is
redirected to the log file.

\item{\textbf{TVS output file}}\\
A TVS output file name can be supplied with the \textbf{-tvs}
option. In this case the output of the analysis in TVS format is
directed to the file.

\item{\textbf{DOT output file}}\\
A DOT output file name can be supplied with the \textbf{-dot}
option. In this case the output of the analysis in DOT format is
directed to the file. The activation scripts for TVLA use this
option to create an output file for DOT by adding the `.dt'
extension to the program name. This option can be used to specify
a different file name for DOT, but the Postscript output should
then be created manually (the dotps script, supplied with TVLA,
can be used for this).

% This option is no longer supported
%\item{\textbf{Dumping intermediate results}}\\
%Some of the analyses can take a very long time. To see
%intermediate results the \textbf{-dump} flag can be used. The
%number supplied is the number of structures generated between
%each dump of the current set of structures per program location.

\item{\textbf{On-line information}}\\
TVLA reports information as the analysis progresses to the
standard output. To avoid seeing this information use the
\textbf{-terse} flag.

\item{\textbf{Preprocessor macros}}\\
Preprocessor macros can be passed to the parsers by using the\\
\textbf{-D$<$macro$>$(value)} option. This has the same effect
as\\
\verb"#define symbol value". For example, specifying -DCONCRETE
defines the CONCRETE symbol, and -DLEVEL(1) sets the value 1 to
the symbol LEVEL.

\item{\textbf{Search path}}\\
The option \textbf{-path} can be used to add directories to the
search path.

%%%%%%%%%%%%%%%%%%%%%%%%%%%%% Moved to property files %%%%%%%%%%%%%%%%%%%%%%%%%%%%
% \item{\textbf{Blur}}\\
% Two rules are implemented when determining which node
% should remain distinguished in the blur.  The default is two nodes must be
% different in the value of at least one abstraction predicate.  The second is two
% nodes must be different in a \emph{definite} value of at least one abstraction
% predicate.  Use the \textbf{-b2} flag to use the second rule.
%%%%%%%%%%%%%%%%%%%%%%%%%%%%%%%%%%%%%%%%%%%%%%%%%%%%%%%%%%%%%%%%%%%%%%%%%%%%%%%%%%

%%%%%%%%%%%%%%%%%%%% -significant is no longer supported %%%%%%%%%%%%%%%%%%%%%%%%%
%\item{\textbf{Node names}}\\
%It may be useful to track the nodes themselves
%from the original structures and during the different actions. Use the
%\textbf{-significant} to try and keep the names of the nodes significant. A
%materialization creates two nodes \param{node}.0 and \param{node}.1 according to
%the focused value of the predicate in question.  A blur that causes merging of
%nodes names the new nodes [\param{node},\ldots,\param{node}].  However, the
%nodes allocated by the new operation new node names in the
%form m\param{number}. In complex analyses using significant names may create
%very large names that create unreadable graphs.
%%%%%%%%%%%%%%%%%%%%%%%%%%%%%%%%%%%%%%%%%%%%%%%%%%%%%%%%%%%%%%%%%%%%%%%%%%%%%%%%%%
\end{itemize}
