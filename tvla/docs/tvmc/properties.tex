\section{Property Verification}

\subsection{Safety Properties}

Verification of safety properties is performed by exploring the
state-space, identifying configurations that may violate the
property requirements.

\figref{Explore} shows a depth-first search algorithm for
exploring a state-space. For each configuration $C$ such that $C$
is not already a \emph{memeber} of the \emph{state-space}, we
explore every configuration $C'$ that can be produced by applying
some action to the current configuration $C$. Note that the
$verify$ step checks if the current configuration $C$ satisfies
the property requirements.

\begin{figure}
\begin{center}
\framebox{ \vbox{
\begin{alltt}
\begin{tabbing}

init\=ialize($C_{0}$) \{ \+ \\
    for \=each $C \in C_{0}$ \+ \\
        push(stack,C) \- \- \\
\} \\


ex\=plore() \{ \+ \\
  wh\=ile stack is not empty \{ \+ \\
    $C$ = pop(stack) \\
    \emph{verify(C)}     \\
    if \= not $member(C,stateSpace)$ \{ \+ \\
      $stateSpace' = stateSpace \cup \{ C \} $ \\
      fo\=r each action $ac$ \+ \\
        fo\=r each $C'$ such that $ C \Rewrites_{ac} C'$  \+ \\
            push(stack,$C'$) \- \- \- \\
    \} \- \\
  \} \- \\
\}

\end{tabbing}
\end{alltt}
} }
\end{center}
\caption{\label{Fi:Explore}State space exploration.}
\end{figure}


The $verify$ step actually executes all \emph{global actions}
defined by the user. The \emph{global actions} can be used issue
messages or to halt analysis when a property is violated.

\subsection{LTL Properties}

$\tvmc$ supports verification of general LTL properties by using
Buchi automata to represent the (negation of the) property to be
verified.

$\tvmc$ takes as input a \texttt{BUC} file that models the Buchi
automaton for the negation of the property to be verified.
Currently, $\tvmc$ does not support automatic construction of the
Buchi automaton from an LTL formula. The Buchi automaton should be
supplied by the user using the \texttt{BUC} file.

A \texttt{BUC} file consists of two parts - the first section of
the file defines the instrumentation predicates to be used as the
vocabulary of the automaton, the second section defines automaton
states and transitions (and the initial state). The sections are
separated by a special separator (\%\%).

Vocabulary instrumentation predicates are nullary predicates
defined in the same manner as in the TVM file.

Automaton is described using transition as seen in
\figref{bucFile}.

\begin{figure}
\begin{center}
\framebox{ \vbox{
\begin{alltt}
\begin{tabbing}

\%i i\_1() = E(v) isthread(v) \\
\%i i\_2() = E(v) E(l) isthread(v) \& blocked(v,l) \\
\%\% \\
\%b\=uchi example \{ \+ \\
    ->(b1) \\
    (b1) i\_1 b2 \\
    b2 i\_2 b3 \\
    b3 i\_1 b4 \\
    b4 i\_2 b5 \\
    b5 i\_1 (b1) \- \\
    \} \\
\end{tabbing}
\end{alltt}
} }
\end{center}
\caption{\label{Fi:bucFile}BUC file defining a Buchi automata.}
\end{figure}


Accepting states of the Buchi automaton are written as $(s)$, an
accepting state should always appear inside parentheses. The
initial state of the automaton should be defined in the first line
of automaton description using $\rightarrow$. Other transitions
are defined by triples of state-label-state, where the label
corresponds to one of the vocabulary predicates.
