\section{Property files}
\label{Se:PropertyFiles}

A mechanism alternative to command-line options is available by
the means of property files. Property files consist of key-value
pairs using the syntax key=value. (More information about the
syntax of property files is available from the JDK documentation
of the java.util.Properties
class\footnote{\url{http://java.sun.com/j2se/1.4.2/docs/api/java/util/Properties.html}}.)
Each command-line option has a corresponding property.

There are several ways to pass properties to TVLA:
\begin{enumerate}
\item When TVLA is activated a tvla.properties file, which is
located at the installation directory is loaded. This file
contains all available properties and also includes documentation
for each.

\item Though it is possible to edit the tvla.properties file, it
is not recommended, because later TVLA versions may update this
file and also because this global change will affect every run of
TVLA.
%
If global changes are intended, add the properties that should be
change to the user.properties file, which is also located at the
installation directory. The properties in this file override the
ones in the tvla.properties file.

\item By creating a file with the same name as the program file
and the .properties extension (for example merge.properties for
merge.tvp) the properties in this file will load when this program
is used and override the ones passed by the previous methods.

\item By specifying a properties file name with the -props
command-line option. The properties in this file will override all
properties specified with the previous methods.

%\item By passing properties from the Java invocation line of TVLA.
%These properties will be overridden by properties specified by all
%other methods.

\item Finally, command-line options change their corresponding
properties and override properties specified by all other methods.
\end{enumerate}

Although all command-line options have corresponding properties,
there are other properties which do not correspond to command-line
options.  These are used to control TVLA behaviors which are less
common and also to test features as they are being developed.
